\begin{description} \itemsep1pt \parskip1pt \parsep0pt
    \item[labMT] --- language assessment by Mechanical Turk \cite{dodds2015human}.
    \item[ANEW] --- Affective Norms of English Words \cite{bradley1999affective}.
    \item[LIWC07] --- Linguistic Inquiry and Word Count, three version \cite{pennebaker2001linguistic}.
    \item[MPQA] --- The Multi-Perspective Question Answering (MPQA) Subjectivity Dictionary \cite{wilson2005recognizing}.
    \item[OL] --- Opinion Lexicon, developed by Bing Liu) \cite{liu2010sentiment}.
    \item[WK] --- Warriner and Kuperman rated words from SUBTLEX by Mechanical Turk \cite{warriner2013norms}.
    \item[LIWC01] --- Linguistic Inquiry and Word Count, three version \cite{pennebaker2001linguistic}.
    \item[LIWC15] --- Linguistic Inquiry and Word Count, three version \cite{pennebaker2001linguistic}.
    \item[PANAS-X] --- The Positive and Negative Affect Schedule --- Expanded \cite{watson1999panas}.
    \item[Pattern] --- A web mining module for the Python programming language, version 2.6 \cite{de2012pattern}.
    \item[SentiWordNet] --- WordNet synsets each assigned three sentiment scores: positivity, negativity, and objectivity \cite{baccianella2010sentiwordnet}.
    \item[AFINN] --- Words manually rated -5 to 5 with impact scores by Finn Nielsen \cite{nielsen2011new}.
    \item[GI] --- General Inquirer: database of words and manually created semantic and cognitive categories, including positive and negative connotations \cite{stone1966general}.
    \item[WDAL] --- Whissel's Dictionary of Affective Language: words rated in terms of their Pleasantness, Activation, and Imagery (concreteness) \cite{whissell1986dictionary}.
    \item[EmoLex] --- NRC Word-Emotion Association Lexicon: emotions and sentiment evoked by common words and phrases using Mechanical Turk \cite{mohammad2013crowdsourcing}.
    \item[MaxDiff] --- NRC MaxDiff Twitter Sentiment Lexicon: crowdsourced real-valued scores using the MaxDiff method \cite{kiritchenko2014sentiment}.
    \item[HashtagSent] --- NRC Hashtag Sentiment Lexicon: created from tweets using Pairwise Mutual Information with sentiment hashtags as positive and negative labels (here we use only the unigrams) \cite{zhu2014nrc}.
    \item[Sent140Lex] --- NRC Sentiment140 Lexicon: created from the ``sentiment140'' corpus of tweets, using Pairwise Mutual Information with emoticons as positive and negative labels (here we use only the unigrams) \cite{MohammadKZ2013}.
    \item[SOCAL] --- Manually constructed general-purpose sentiment dictionary \cite{taboada2011lexicon}.
    \item[SenticNet] --- Sentiment dataset labelled with semantics and 5 dimensions of emotions by Cambria \etal, version 3 \cite{cambria2014senticnet}.
    \item[Emoticons] --- Commonly used emoticons with their positive, negative, or neutral emotion \cite{gonccalves2013comparing}.
    \item[SentiStrength] --- an API and Java program for general purpose sentiment detection (here we use only the sentiment dictionary) \cite{thelwall2010sentiment}.
    \item[VADER] --- method developed specifically for Twitter and social media analysis \cite{hutto2014vader}.
    \item[Umigon] --- Manually built specifically to analyze Tweets from the sentiment140 corpus \cite{levallois2013umigon}.
    \item[USent] --- set of emoticons and bad words that extend MPQA \cite{pappas2013distinguishing}.
    \item[EmoSenticNet] --- extends SenticNet words with WNA labels \cite{poria2013enhanced}.
  \end{description}
